 \documentclass[11pt, oneside]{article} 
\usepackage{geometry}
\geometry{letterpaper} 
\usepackage{graphicx}
	
\usepackage{amssymb}
\usepackage{amsmath}
\usepackage{parskip}
\usepackage{color}
\usepackage{hyperref}

\graphicspath{{/Users/telliott_admin/Dropbox/Tex/png/}}
% \begin{center} \includegraphics [scale=0.4] {gauss3.png} \end{center}

\title{Area formula by geometry}
\date{}

\begin{document}
\maketitle
\Large
We use basic geometry to derive the two formulas for the area of inscribed and circumscribed polygons from the basic geometry.

The area formulas for inside ($a$) and outside ($A$) polygons are those for a circle of unit radius (so that $\pi$ is the area):
\[ A' = 2 \frac{a'A}{a' + A} \]
\[ a' = \sqrt{aA} \]

However, we need another symbol for area, because $a$ is also currently the line segment corresponding to $p/n$.  Let's use $I$ and $C$ for the inside and outside areas, to match the source.  

\url{https://divisbyzero.com/2018/09/28/proof-without-word-gregorys-theorem/}

We will also adopt their $n$ and $2n$ notation, It's a bit clumsy but that will make it easier to match things up.
\[ C_{2n} = 2 \cdot \frac{I_{2n} C_n}{I_{2n} + C_{n}} \]
\[ I_{2n} = \sqrt{I_n C_n} \]

The first two areas are $I_n$ and $I_{2n}$
\begin{center} 
\includegraphics [scale=0.3] {Gregory1.png} 
\includegraphics [scale=0.3] {Gregory2.png} 
\end{center}
We compute these areas for the whole sector of angle $\theta$, so there are two congruent triangles with base $a$ (or base $r$) and height $c$. Multiply by $n$ if you like to get the entire polygon, but every expression will have a factor of $n$, and we'll be looking at ratios, so we could just not worry about it.

The third easy one is $C_n$:
\begin{center}
\includegraphics [scale=0.3] {Gregory3.png}
\includegraphics [scale=0.3] {Gregory4.png} 
 \end{center}
 
We write the last one ($C_{2n}$) as two different differences.
\begin{center} 
\includegraphics [scale=0.3] {Gregory5.png} 
\includegraphics [scale=0.3] {Gregory6.png} 
\end{center}

Let's gather all these expressions in one place, forming ratios:
\[ \frac{I_{2n}}{I_n} = \frac{ncr}{nca} = \frac{r}{a} \]
\[ \frac{C_n}{I_{2n}} = \frac{ncb}{ncr} = \frac{b}{r}  \]
\[ \frac{C_n - C_{2n}}{C_{2n} - I_{2n}} = \frac{n(b-r)d}{n(r-a)d} = \frac{b-r}{r-a} \]

We will prove that these three ratios are all equal to each other.  

We will have used the geometry to prove what the source calls their Lemmas, and those can be used in turn to prove the original Gregory formulas.

But the proof is easy:
\begin{center} \includegraphics [scale=0.5] {Gregory7.png} \end{center}

It's just a matter of similar triangles:
\[ \frac{r}{a} = \frac{b}{r} = \frac{b-r}{r-a} \]

That's the "without words" part.

For that very last part, you can work out the dimensions of the tiny similar triangle, or you can say:
\[ \frac{r}{a} = \frac{b}{r} \]
\[ \frac{r}{a} - \frac{a}{a} = \frac{b}{r}- \frac{r}{r} \]
\[ \frac{r-a}{a} = \frac{b-r}{r} \]
which is easily rearranged to give the desired result.

$\square$

This can also be proved using the \hyperref[sec:angle_bisector]{\textbf{angle bisector theorem}}.
\begin{center} \includegraphics [scale=0.25] {Gregory10.png} \end{center}
The side labeled $e$ bisects the angle formed by the two sides labeled $c$ and $f$.  

Therefore, by a corollary of the angle bisector theorem:
\[ \frac{b-r}{f} = \frac{r-a}{c} \ \ \Rightarrow \ \  \frac{b-r}{r-a} = \frac{f}{c} \]
But $f$ and $c$ are two sides of a triangle which is similar to the colored portions below:

\begin{center} 
\includegraphics [scale=0.25] {Gregory3.png} 
\includegraphics [scale=0.25] {Gregory1.png} 
\end{center}
Therefore
\[ \frac{b}{r} = \frac{r}{a} = \frac{f}{c} = \frac{b-r}{r-a}  \]
As we said.

\subsection*{algebra}
Moving on to the geometric mean formula is not hard.  From above we have that
\[ \frac{I_{2n}}{I_n} = \frac{C_n}{I_{2n}}   \]
\[ [I_{2n}]^2 = I_n C_n  \]
Translated back into the $A,a$ area notation
\[ a' = \sqrt{aA} \]
This is just what we wanted to show.

For the other formula, what we have is:
\[ \frac{C_n - C_{2n}}{C_{2n} - I_{2n}} = \frac{C_n}{I_{2n}}   \]
\[ I_{2n} (C_n - C_{2n}) = C_n (C_{2n} - I_{2n}) \]
\[ 2 I_{2n} C_n = C_n C_{2n} + I_{2n} C_{2n} \]
\[ = C_{2n}(C_{n} + I_{2n}) \]
(It took me a while to figure this part out.  It helps to label everything with single letters when pushing symbols around).

So
\[ C_{2n}  = 2 \cdot \frac{I_{2n} C_n} {C_{n} + I_{2n}}  \]
\[ C_{2n} = 2 \cdot \frac{1}{1/I_{2n} + 1/C_n} \]

And we're done.  In our preferred notation
\[ A' = 2 \cdot \frac{1}{1/a' + 1/A} \]

\subsection*{historical note}

The area-based formulas given above are due to James Gregory.

\url{https://divisbyzero.com/2018/09/28/proof-without-word-gregorys-theorem/}

As an aside, the Fundamental Theorem of Calculus (FTC) is usually thought about using the language of functions, and ascribed mainly to Leibnitz, with some credit to the two Isaacs, Newton and his university lecturer, Barrow.

\url{https://arxiv.org/abs/1111.6145}

Amazingly enough, Gregory published a geometric (Euclidean) proof of the FTC in 1668.  That predates Liebnitz (1693) by more than 25 years!  

This should motivate us to give considerable credit to individuals other than Newton and Liebnitz, including Fermat, Pascal, Wallis, Gregory and more, in the invention of the calculus.


\end{document}