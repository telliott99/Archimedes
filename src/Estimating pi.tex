\documentclass[11pt, oneside]{article} 
\usepackage{geometry}
\geometry{letterpaper} 
\usepackage{graphicx}
	
\usepackage{amssymb}
\usepackage{amsmath}
\usepackage{parskip}
\usepackage{color}
\usepackage{hyperref}

\graphicspath{{/Users/telliott_admin/Dropbox/Tex/png/}}
% \begin{center} \includegraphics [scale=0.4] {gauss3.png} \end{center}

\title{Estimating pi}
\date{}

\begin{document}
\maketitle
\Large
We can get an estimate for the value of $\pi$ by looking at the perimeters for the series square, hexagon, octagon.  The circle has diameter of $1$, so the value of $\pi$ lies between the internal and external perimeters.

\subsection*{square}
\begin{center} \includegraphics [scale=0.35] {pi_sq.png} \end{center}
Dividing the internal square along the diagonal, the right triangle is isosceles with hypotenuse equal to $1$ and sides equal to $1/\sqrt{2}$.  The internal perimeter is $4/\sqrt{2} = 2 \sqrt{2} = 2.828$.

The external perimeter is easy, it is just $4$ times the diameter, or $4$.  

\subsection*{hexagon}
\begin{center} \includegraphics [scale=0.35] {pi_hex.png} \end{center}
Each of the six sectors of a hexagon consists of an equilateral triangle, with side equal to $1/2$, so the internal perimeter is $6 \cdot 1/2 = 3$.  

The six sectors of the external perimeter each have an altitude equal to $1/2$.  The half-sectors are $30-60-90$ triangles with sides $1-2-\sqrt{3}$.  Hence the ratio of one-half an external side to the radius is $1/\sqrt{3}$ and since the radius is $1/2$, the actual value is $1/2 \sqrt{3}$.  So each whole side is $1/\sqrt{3}$, giving $6/ \sqrt{3} = 3.464$.

\subsection*{octagon}
\begin{center} \includegraphics [scale=0.35] {pi_oct.png} \end{center}
The octagon suddenly becomes quite challenging.  I really don't know any way to do this except by using trigonometry, and the corollary of the angle bisector rule that says:
\[ \cot \theta + \csc \theta = \cot \theta/2 \]
For a 45 degree angle the cotangent is $1$ and the cosecant is $\sqrt{2}$ so
\[ \cot \theta/2 = 1 + \sqrt{2} \]

Hence for the external perimeter, let the half-side be $h$, then
\[ \frac{r}{h} = \cot \theta/2 = 1 + \sqrt{2} \]
\[ h = \frac{1/2}{1 + \sqrt{2}} \]
and there are sixteen of them so
\[ P = \frac{8}{1 + \sqrt{2}} = 3.3137 \]

For the internal perimeter, we have the hypotenuse, hence we will need the cosecant.  Label the sides of a right triangle $a,b,c$.  The Pythagorean theorem:
\[ a^2 + b^2 = c^2 \]
\[ \frac{a^2}{b^2} + 1 = \frac{c^2}{b^2} \]
If we choose $a$ as the adjacent side, then $a/b$ is the cotangent and $c/b$ is the cosecant.  Hence
\[ \csc \theta = \sqrt{\cot^2 \theta + 1} \]
and in our case
\[ \csc \pi/ 8 = \sqrt{ (1 + \sqrt{2})^2 + 1 } = 2.613 \]

Let $h$ be the half-side then
\[ \frac{1/2}{h} = \csc \theta = 2.613 \]
\[ h = 0.19 \]
The total perimeter is $16$ times this or $3.0615$

This is round 1 of Archimedes calculation.

\end{document}
