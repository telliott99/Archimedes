\documentclass[11pt, oneside]{article} 
\usepackage{geometry}
\geometry{letterpaper} 
\usepackage{graphicx}
	
\usepackage{amssymb}
\usepackage{amsmath}
\usepackage{parskip}
\usepackage{color}
\usepackage{hyperref}

\graphicspath{{/Users/telliott_admin/Dropbox/Tex/png/}}
% \begin{center} \includegraphics [scale=0.4] {gauss3.png} \end{center}

\title{Euclid's proof of angle bisector corollary}
\date{}

\begin{document}
\maketitle
\Large
Consider the following $\triangle ABC$, with $\angle BAC$ bisected by $AE$.
\begin{center} \includegraphics [scale=0.3] {angle_bisector5.png} \end{center}
The angle bisector theorem says that $AB:BD = AC:DC$, and the corollary says that $AB + AC: BC$ is the same ratio.
\subsection*{Euclid's proof}
Euclid's proof is Book 6, proposition 3.
\begin{center} \includegraphics [scale=0.3] {angle_bisector6.png} \end{center}
Draw $CE$ parallel to $DA$.  Then, $\angle DAC = \angle ACE$ (by alternate interior angles).  

We are given that $\angle DAB = \angle CAD$, and now see that $\angle AEC$ and $\angle DAB$ are equal as corresponding angles of a line cutting two parallel lines.  Therefore all four angles (marked $\phi$) are equal.

Therefore, $\triangle ACE$ is isosceles, with $AC = AE$.

Furthermore, $\triangle ABD$ is similar to $\triangle EBC$.  Therefore, $AB:BD = EB:BC$.  

And since $EB = EA + AB$ and $AC = AE$, we have that $AB:BD = AB + AC:BC$.

\subsection*{reverse}
This proof just reverses each statement above.  We are given $AB:AC = BD:DC$.  

Both are equal to $AB:AE$, by similar triangles, so $AC = AE$.  

Therefore $\triangle ACE$ is isosceles with $\angle ACE  = \angle AEC$.

But $\angle AEC = \angle BAD$ (isosceles)

and $\angle ACE = \angle CAD$ (alternate interior angles)

so $\angle CAD = \angle BAD$.

$\square$

\end{document}