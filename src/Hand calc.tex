\documentclass[11pt, oneside]{article} 
\usepackage{geometry}
\geometry{letterpaper} 
\usepackage{graphicx}
	
\usepackage{amssymb}
\usepackage{amsmath}
\usepackage{parskip}
\usepackage{color}
\usepackage{hyperref}

\graphicspath{{/Users/telliott_admin/Tex/png/}}
% \begin{center} \includegraphics [scale=0.4] {gauss3.png} \end{center}

\title{Hand calculating}
\date{}

\begin{document}
\maketitle
\Large

\subsection*{analysis of the calculations:  perimeter}

$\circ$  base case is a square

$\circ$  perimeter of circle with $d = 1$

$\circ$  rule: do $P$ first then $p$

formulas

\[ P' = 2pP/(p + P) \]
\[ p' = \sqrt{pP'} \]

initialization

\[ P = 4 \]
\[ p = 2 \sqrt{2} \]

Let's switch variables to make the general case clear.  Let $x = P = 4, y = p = 2 \sqrt{2}$ and follow the rules above:

\[ x' = 2xy/(x+y) \]

\[ y' = \sqrt{x'y} \]

\[ x = x'; y = y' \]
\[ x' = 2xy/(x+y) \]

\subsection*{Area}

$\circ$  base case is still a square

$\circ$  area of circle with r = 1

$\circ$  rule:  do $a$ first then $A$

formulas

\[ a' = \sqrt{aA} \]
\[ A' = 2a'A(a'+A) \]

initialization

\[ a = 2 \] 
\[ A = 4 \]

a special first step for area

\begin{verbatim}
a' = sqrt(aA)
   = 2 sqrt{2)
a = a'
\end{verbatim}

Now, let $x = A = 2, y = a = 2 \sqrt{2}$.  Note the switched order, lower case $a$ second.

\[ x' = 2xy/(x+y) \]

\[ y' = \sqrt{x'y} \]

\[ x = x'; y = y' \]
\[ x' = 2xy/(x+y) \]

It's exactly the same calculation!  The change in order of operations is seen to be due to the fact that we need a first step for the area, to convert $a = 2$ to $a = 2 \sqrt{2}$.

Let's actually do a few steps of the calculation by hand.

\subsection*{perimeter}
$P$ first, then $p$.  For a square on a circle of diameter $1$, the outside perimeter is
\[ P = 4 \]
while the inside is
\[ p = 2 \sqrt{2} \]

Next:  $P$ (re-using the symbol):
\[ P = \frac{2pP}{p+P} = \frac{16 \sqrt{2}}{4 + 2 \sqrt{2}} = \frac{8}{1 + \sqrt{2}} = 3.3137 \]
Then $p$:
\[ p = \sqrt{pP} = \sqrt{2 \sqrt{2} \cdot \frac{8}{1 + \sqrt{2}}} = 4 \sqrt{\frac{1}{1 +1/ \sqrt{2}}} = 3.0615  \]

\subsection*{area}
$a$ first, then $A$.  For a square on a circle of radius $1$, the inside area is
\[ a = \sqrt{2} \cdot \sqrt{2} = 2 \]
while the outside is 
\[ A = 4 \]
Next (re-using the symbol):
\[ a = \sqrt{aA} = \sqrt{8} = 2 \sqrt{2} \]
Then $A$
\[ A = \frac{2aA}{a + A} = 16 \frac{\sqrt{2} }{4 + 2 \sqrt{2}} = \frac{8}{1 + \sqrt{2}} = 3.3137 \]
Then $a$:
\[ a = \sqrt{aA} = \sqrt{2 \sqrt{2} \cdot \frac{8}{1 + \sqrt{2}}} = 4 \sqrt{\frac{1}{1 +1/ \sqrt{2}}} = 3.0615  \]

The calculations are identical, and will remain so from here on out.

\subsection*{sine and tangent}
$n=4$
\[ n \sin \theta = 4 \cdot \frac{1}{\sqrt{2}} = 2 \sqrt{2} \]
\[ n \tan \theta = 4 \cdot 1 = 4 \]
At this stage, the inner area by trigonometry is equal to the second value of $a$ above, and the outer area is equal to the first value of $A$ above.

Calculate $S'$, etc:
\[ C' = \sqrt{\frac{1}{2} (1 + C)} = \sqrt{\frac{1}{2} (1 + 1/\sqrt{2}) } \]
\[ S' = S/2C' = \frac{1}{2 \sqrt{2}} \cdot \sqrt{\frac{2}{1 + 1/\sqrt{2}}} = \frac{1}{2} \cdot \sqrt{\frac{1}{1 + 1/\sqrt{2}}} \]
\[ T' = \frac{S}{1 + C} = \frac{1/\sqrt{2}}{1 + 1/\sqrt{2}} = \frac{1}{1 + \sqrt{2}} \]

so for $n=8$
\[ n S' = 4 \sqrt{\frac{1}{1 + 1/\sqrt{2}}} \]
\[ n T' = \frac{8}{1 + \sqrt{2}} \]
It continues.  At this stage, the inner area is equal to the third value of $a$ above, while the outer area is equal to the second value of $A$ above.

It's the same calculation, in disguise.

\end{document}