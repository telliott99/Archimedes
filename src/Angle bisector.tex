\documentclass[11pt, oneside]{article} 
\usepackage{geometry}
\geometry{letterpaper} 
\usepackage{graphicx}
	
\usepackage{amssymb}
\usepackage{amsmath}
\usepackage{parskip}
\usepackage{color}
\usepackage{hyperref}

\graphicspath{{/Users/telliott_admin/Dropbox/Tex/png/}}
% \begin{center} \includegraphics [scale=0.4] {gauss3.png} \end{center}

\title{Angle bisector}
\date{}

\begin{document}
\maketitle
\Large

Let's review some ideas related to angle bisection.  Recall that if we have an angle bisector in a right triangle, the theorem says that 
\begin{center} \includegraphics [scale=0.4] {angle_bisector4.png} \end{center} 
\[ \frac{OA}{OC} = \frac{AD}{DC}  \]
In slightly different language, we label the sides with letters:
\begin{center} 
\includegraphics [scale=0.3] {pi3.png} 
\end{center}
\[ \frac{a}{c} = \frac{b}{d} \]

Proof: draw the given altitude of the top triangle, forming two congruent triangles plus a smaller one. As always, construction of the altitude gives a triangle similar to the original.
\begin{center} 
\includegraphics [scale=0.3] {pi4.png} 
\end{center}
For example, by complementary angles, the angle between the sides labeled $c$ and $d$ is $2 \theta$.  Therefore this small triangle is similar to the original one.  

By similar triangles, then, we have 
\[ \frac{d}{c} = \frac{b}{a} \]
which can be rearranged to give the desired statement.  A corollary follows:
\[ \frac{a}{b} = \frac{c}{d} \]
\[ \frac{a + b}{b} = \frac{c + d}{d} \]
\[ \frac{a + b}{c + d} = \frac{b}{d} = \frac{a}{c} \]
\begin{center} \includegraphics [scale=0.3] {pi3.png} \end{center}
This doesn't seem obvious to me, in fact it seems counter-intuitive.  Nonetheless, we use it extensively.

\subsection*{application}

In the write-up that follows Archimedes in using line segments we said:

Let $AD$ bisect the angle ($\angle BAC$), and then join $BD$.
\begin{center} \includegraphics [scale=0.4] {pi7.png} \end{center}
$\bullet$  $\angle BAD = \angle dAC = \angle dBD$.

The first statement just restates the construction as an angle bisector.  The second follows from the fact that the two angles have vertices on the circle and cut off the same arc.  

As a consequence, $\triangle dBD \sim \triangle dAC$.

Start with the similar triangles above and write three ratios of long side (not hypotenuse) to short side
\[ AD : BD = BD : Dd = AC : Cd \]
Note: the source has $AB:Bd$ but this seems to be an error.  That is a ratio of two hypotenuses and so is not equal to the others.  As a result, I was unable to follow this part of the proof:
\begin{center} \includegraphics [scale=0.6] {pi8.png} \end{center}

However, I was able to prove the last statement
\[ (AB + AC):BC = AD:DB \]

The proof is as follows.
\begin{center} \includegraphics [scale=0.4] {pi7.png} \end{center}
We have that $\triangle ABC$ is a right triangle and that $AD$ and thus $Ad$ is the angle bisector for $\angle BAC$.  Therefore, we have by our favorite theorem that
\[ (AB + AC):BC =  AC:Cd \]
We also have that $\triangle ABD$ is a right triangle and by virtue of the angle bisector construction, $\triangle ABD$ is similar to $\triangle ACd$.  Therefore:
\[ AC:Cd = AD:DB \]
These two lines combine to give the desired result.  

In the modified write-up where I used simpler labels for the sides, I just said this:

we redraw [the diagram] as
\begin{center} \includegraphics [scale=0.4] {pi10.png} \end{center}
Once again we will need the cotangent and the cosecant.  For the largest triangle, a right triangle, these are
\[ \cot \theta = \frac{c}{d + e} \]
\[ \csc \theta =  \frac{a}{d + e} \]
We have, by the angle bisector corollary
\[ \cot \theta + \csc \theta =  \frac{a + c}{d + e} = \frac{c}{d} = \cot \theta/2 \]
 
We have $c/d$ and still need $b/d$.  But the Pythagorean theorem says that $c^2 + d^2 = b^2$ so
\[ \frac{b^2}{d^2} = \frac{c^2}{d^2} + 1 \]
\[ \csc \theta/2 = \sqrt{\cot^2 \theta/2 + 1} \]

\end{document}