\documentclass[11pt, oneside]{article} 
\usepackage{geometry}
\geometry{letterpaper} 
\usepackage{graphicx}
	
\usepackage{amssymb}
\usepackage{amsmath}
\usepackage{parskip}
\usepackage{color}
\usepackage{hyperref}

\graphicspath{{/Users/telliott_admin/Tex/png/}}
% \begin{center} \includegraphics [scale=0.4] {gauss3.png} \end{center}

%break
\title{Double and half angles}
\date{}

\begin{document}
\maketitle
\Large

\label{sec:double_half_angles}

We will find it useful in several problems to be able to compute the sine, cosine and tangent of angle $2\theta$, knowing the values of the functions for angle $\theta$.  These formulas can be rearranged to give the values of $\theta/2$ in terms of $\theta$.

I usually can't remember these formulas, but easily derive them from the sum of angles when needed.

\subsection*{cosine}

Start with our old friend:

\[ \cos s + t = \cos s \cos t - \sin s \sin t \]

Let $s=t$:
\[ \cos 2s = \cos^2 s - \sin^2 s \]
Since $\sin^2 s + \cos^2 s = 1$, $-\sin^2 = \cos^2 - 1$ so
\[  \cos 2s = 2 \cos^2 s - 1 \]

We can use this formula to compute the value for $2s$ given that for $s$.  To go from $2\theta$ to $\theta$:
\[ \cos^2 s = \frac{1}{2}(1 + \cos 2s) \]
\[ \cos s = \sqrt{ \frac{1}{2}(1 + \cos 2s)} \]

\subsection*{sine}
\[ \sin s + t = \sin s \cos t + \cos t \sin s \]

Let $s = t$:
\[ \sin 2s = 2 \sin s \cos s \]

Put the other way
\[ \sin s = \frac{\sin 2s}{2 \cos s} \]

\subsection*{tangent}

The formulas for the tangent are easily obtained by substitution.  Let us simplify the notation a bit by setting $S = \sin 2t$ and $S' = \sin t$ and similarly for cosine and tangent.  From above we have the basic relationships

\[ S' = \frac{S}{2 C'} \]
and
\[ C' = \sqrt{\frac{1}{2} (1 + C)}  \]
\[ 2[C']^2 = 1 + C \]

So the tangent ($T' = \tan s$) is:
\[ T' = \frac{S'}{C'} = \frac{S}{2C'} \ \frac{1}{C'} = \frac{S}{2 [C']^2} \]
\[ = \frac{S}{1 + C} \]
That's a fairly remarkable simplification!

Another way to say the same thing:
\[ \frac{1}{T'} = \frac{1}{T} + \frac{1}{S} \]

The last version is used extensively by Archimedes, although it isn't labeled as such.  Another way to say it in modern language is
\[ \cot \theta/2 = \cot \theta + \csc \theta \]

This comes from our half-angle formula and is also proven easily using the angle bisection theorem.

This result can be massaged in various ways.  Multiply on the top and bottom of the right-hand side by $T$
\[ T' = \frac{ST}{S + T} \]

Also, since
\[ T'  = \frac{S}{1 + C} = \frac{S'}{C'} \]
\[ C' = \frac{S'(1+C)}{S} \]

In going from unprimed ($2 \theta$) to prime ($\theta$), it seems that the most straightforward way is to compute one or both of these first

\[ C' = \sqrt{\frac{1}{2} (1 + C)}  \]
\[ T'  = \frac{S}{1 + C} \]
and then the sine requires $C'$ so it comes later.
\[ S' = \frac{S}{2 C'} \]


\end{document}