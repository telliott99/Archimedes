\documentclass[11pt, oneside]{article} 
\usepackage{geometry}
\geometry{letterpaper} 
\usepackage{graphicx}
	
\usepackage{amssymb}
\usepackage{amsmath}
\usepackage{parskip}
\usepackage{color}
\usepackage{hyperref}

\graphicspath{{/Users/telliott_admin/Dropbox/Tex/png/}}
% \begin{center} \includegraphics [scale=0.4] {gauss3.png} \end{center}

\title{Perimeter formula}
\date{}

\begin{document}
\maketitle
\Large

This web page originally got me started with this derivation

\url{http://personal.bgsu.edu/~carother/pi/Pi3d.html}

(Unfortunately, the link is dead now, probably because the University took Dr. Carother's pages down when he died, idiots).  It has been preserved by the wayback machine:

\url{https://web.archive.org/web/20171024182015/http://personal.bgsu.edu/~carother/pi/Pi3d.html}

On that page, there was given a simple pair of formulas listed, namely, for an inside perimeter $p$ and an outside perimeter $P$
\[ P' = \frac{2pP}{p + P} \]
\[ p' = \sqrt{pP'} \]

The first equation can be rearranged to give
\[ \frac{1}{P'} = \frac{1}{2} \ [ \frac{1}{P} + \frac{1}{p} \ ] \]
which is the definition of the harmonic mean of $p$ and $P$, while the second equation is the geometric mean.

Since in our derivation $p$ and $P$ are the same multiple of $S$ and $T$, it seems like the same relationships should hold for the sine and tangent, but we must remember the extra factor of $2$.

From the half-angle formulas, we said that
\[ T'  = \frac{S}{1 + C} \]
Multiply top and bottom on the right by $T$:
\[ T' = \frac{ST}{T + S} \]

Recall that $S$ is the same as $p$, within a factor of $n$, and that $T$ is the same as $P$, within the same factor.
\[ p = nS \]
\[ P = nT \]
while 
\[ P' = 2nT' \]

Going back to 
\[ T' = \frac{ST}{T + S} \]
\[ 2nT' = \frac{2 \cdot nS \cdot nT}{nT + nS} \]
\[ P' = \frac{2pP}{p + P} \]

This is what was given.

For the second one
\[ S' = \frac{S}{2 C'} \]
\[ = \frac{S}{2} \ \frac{T'}{S'} \]
Then
\[ 4[S']^2 = S \cdot 2T' \]
\[ [2nS']^2 = nS \cdot 2nT' \]
Changing  variables, $p' = 2nS'$
\[ [p']^2 = pP' \]

Finally
\[ p' = \sqrt{pP'} \]
which matches what was given.

\end{document}